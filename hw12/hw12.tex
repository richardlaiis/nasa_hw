\documentclass[12pt]{article}
\usepackage{graphicx}
\usepackage[margin=2cm, a4paper]{geometry}
\usepackage{setspace}
\usepackage{pdfpages}
\usepackage{float}
\usepackage{ctex}
\usepackage{amsmath}
\usepackage{fancyvrb}
\usepackage{amssymb}
\usepackage{minted}
\usepackage{enumitem}
\usepackage[colorlinks,linkcolor=blue]{hyperref}

\usepackage{xeCJK}
\setCJKmainfont{Noto Serif TC}

\renewcommand{\contentsname}{Contents}
\renewcommand{\figurename}{Figure}
\renewcommand{\tablename}{Table}
\hypersetup{
    colorlinks=true,
    linkcolor=black,
    filecolor=magenta,      
    urlcolor=blue,
}
\newcommand{\mytitle}{網路管理與系統管理 HW12}
\newcommand{\myauthor}{B13902022 賴昱錡}

\usepackage{fancyhdr}
\pagestyle{fancy}
\fancyhead{}
\fancyhead[L]{\mytitle}
\fancyhead[R]{\myauthor}

\title{\mytitle}
\author{\textbf{\myauthor}}
\date{\today}

\begin{document}

\onehalfspacing
\maketitle

\section*{2.畫中有話}
\subsection*{(a)}
首先我們可以知道 pixels 是 secret\_mygo.png 的像素資料,每一項都是一個長度為 3 的 tuple,分別代表紅、藍、綠的數值 (介於 0-255),在讀入某個字串後 (data),將每一個字元轉為 unicode,再轉為 8 位的 binary string。在處理 pixels 裡面的資料時,這份扣會把連續三格 pixel (pixels[3*i], pixels[3*i+1], pixels[3*i+2]) 變成一個長度為 9 的 list,之後再根據對應到 data 的第 i 個字元所轉成的字串 (方法如前述),改變 list 中的數值 (一個 binary string 有 8 項,也對應到 list 的前八項),最終再將 list 裡面的數值寫回圖片。
\subsection*{(b)}
為了找到藏在圖片裡面的 string,我們可以用 3 個像素為單位來解析,很容易可以發現到,如果 (a) 所述的 binary string 的一項是 "1",則 colors 對應到的一項數值必為奇數,反之則為偶數,所以我們可以跑過足夠多的像素,然後找到很多 binary string 及其對應的 unicode and 字元,用其來拼湊出 flag。

在這裡我使用 python 來找出 flag,程式碼如下,最終得到的結果是: (btw 在那之前可能要安裝 PIL module)

{\centering\verb|HW12{S4KiCh4n_sakiCHAN_S4k1ChaN}|\par}

參考資料是我自己,好耶,這是少數完全不用仰賴 GPT 解出的題目。
\begin{minted}[frame=lines,framesep=2mm,baselinestretch=1.2,linenos,breaklines]{python}
from PIL import Image

image_file = "secret_mygo.png"
img = Image.open(image_file)
pixels = list(img.getdata())
flag = ""

for i in range(0, 100, 3):
    colors = list(pixels[i])+list(pixels[i+1])+list(pixels[i+2])
    tmp = ""
    for j in range(8):
        if colors[j] % 2 == 0:
            tmp = tmp + "0"
        else:
            tmp = tmp + "1"
    flag = flag+chr(int(tmp, 2))

print(flag)
\end{minted}

\section*{4. Introduction to gnireenignE esreveR}
這題的 flag 是:

{\centering\verb|HW12{hW0_8UT_WiTH_r3V3Rse_eN91NE3rinG}|\par}

我拿到 flag 的方法是把 chal.exe 丟進 dogbolt.org,然後發現 Hex-Rays 這個 decompiler 竟然幫我分析好原本程式的架構了。基本上,原本的程式包含以下重要變數:
\begin{minted}[frame=lines,framesep=2mm,baselinestretch=1.2,linenos,breaklines]{c}
char key[9] = "nAs42O2S";
char pattern[40] = {'&','\x16','B','\x06','I','\'','e','c','1','y','&','`','m','\x18'
,'[','\a','&','\x1E','\x01','\a','d','|','`',' ','\v','\x1E','\x16','z','\v','~','|',
'\x16',']','3','\x1A','Z','u','2','\0','\0'}; 
int flag_len = 38; 
\end{minted}

看到程式接受輸入的地方 (應該是 main),他會用 key 對 pattern 做 xor encryption,然後 check 程式接受的參數是不是等於加密後的 pattern。
\begin{minted}[frame=lines,framesep=2mm,baselinestretch=1.2,linenos,breaklines]{c}
int __fastcall main(int argc,
  const char ** argv,
    const char ** envp) {
  int i; // [rsp+1Ch] [rbp-4h]

  if (argc == 2) {
    for (i = 0; i < flag_len; ++i)
      pattern[i] ^= key[i % key_len];
    if (!strcmp(argv[1], pattern))
      puts("Congratulations! You found the flag!");
    else
      puts("Haha! wrong >:)!!!!!!");
    return 0;
  } else {
    puts("Usage: ./chal.exe <flag>");
    return 1;
  }
}
\end{minted}

所以我們可以用前面提到的變數實際去 xor 一遍,就可以得到程式中的 pattern/flag。(將 flag 作為 chal.exe 的參數執行,可以得到代表正確的訊息)
\begin{minted}[frame=lines,framesep=2mm,baselinestretch=1.2,linenos,breaklines]{c}
for (int i = 0; i < flag_len; ++i ) pattern[i] ^= key[i % key_len];
printf("%s\n", pattern);
\end{minted}

\end{document}
% how to display codes?
% \begin{minted}[frame=lines,framesep=2mm,baselinestretch=1.2,linenos,breaklines]{python}
% \end{minted}

% how to display images?
% \begin{figure}[H]
%     \centering
%     \includegraphics[width=0.5\linewidth]{}
%     \caption{Caption}
% \end{figure}
% test